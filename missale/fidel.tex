{\let\cleardoublepage\clearpage\chapter{Missa Fidelium}}

\directio{stans}

\textit{Deinde osculatur altare, et, versus ad populum, dicit:}

\versiculum{Dominus vobiscum.}
\alternatim{\par\responsorium{Et cum spiritu tuo.}}

\textit{Postea dicit}:

\versiculum{Oremus.}

\textit{et antiphonam ad Offertorium.}

\vspace{\baselineskip}

\section{Offertorium}

\directio{sedens}

\proprius{Offertorium}

\textit{%
    Qua dicta, si est Missa solemnis, diaconus porrigit celebranti patenam cum
    hostis: secus sacerdos ipse accipit patenam cum hostia, quam offerens,
    dicit:
}

\initialis{S}uscipe, sancte Pater, omnipotens aeterne Deus, hanc immaculatam
hostiam, quam ego indignus famulus tuus offero tibi Deo meo vivo et vero, pro
innumerabilibus peccatis, et offensionibus, et neglegentiis meis, et pro omnibus
circumstantibus, sed et pro omnibus fidelibus christianis vivis atque defunctis:
ut mihi et illis proficiat ad salutem in vitam aeternam.  Amen.

\textit{%
    Deinde faciens crucem cum eadem patena, deponit hostiam super corporale.
    Diaconus ministrat vinum, subdiaconus aquam in calice: vel si Missa sine
    sacris ministris celebratur, utrumque infundit sacerdos, et aquam miscendam
    in calice benedicit signo crucis, dicens:
}

\initialis{D}eus, qui humanae substantiae dignitatem mirabiliter, condidisti, et
mirabilius reformasti: da nobis, per huius aquae et vini mysterium, eius
divinitatis esse consortes, qui humanitatis nostrae fieri dignatus est
particeps, Iesus Christus; Filius tuus, Dominus noster: Qui tecum vivit et
regnat in unitate Spiritus Sancti Deus: per omnia saecula saeculorum.  Amen.

\divisio

\textit{%
    In Missis defunctorum dicitur praedicta oratio: sed aqua non
    benedicitur.
}

\divisio

\textit{Postea accipit calicem, et offert, dicens}:

\initialis{O}fferimus tibi, Domine, calicem salutaris, tuam deprecantes
clementiam: ut in conspectu divinae maiestatis tuae, pro nostra et totius mundi
salute, cum odore suavitatis ascendat.  Amen.

\textit{%
    Deinde facit signum crucis cum calice, et illum ponit super corporale, et
    palla cooperit: tum, iunctis manibus super altare, aliquantulum inclinatus,
    dicit:
}

\biblia{Dn. 3, 39}

\initialis{I}n spiritu humilitatis et in animo contrito suscipiamur a te,
Domine: et sic fiat sacrificium nostrum in conspectu tuo hodie, ut placeat tibi,
Domine Deus.

\textit{%
    Erectus expandit manus, easque in altum porrectas iungens, elevatis ad
    caelum oculis, et statim demissis, dicit:
}

\initialis{V}eni, sanctificator omnipotens aeterne Deus: (\textit{benedicit
oblata, prosequendo:}) et bene\cross{}dic hoc sacrificium, tuo sancto nomini
praeparatum.

\section{Incensum}

\textit{Postea, si solemniter celebrat, benedicit incensum, dicens}:

\initialis{P}er intercessionem beati Michaelis Archangeli, stantis a dextris
altaris incensi, et omnium electorum suorum, incensum istud dignetur Dominus
bene\cross{}dicere, et in odorem suavitatis accipere.  Per Christum Dominum
nostrum.  Amen.

\textit{%
    Et, accepto thuribulo a diacono, incensat oblata, modo in rubricis
    praescripto, dicens:
}

\initialis{I}ncensum istud a te benedictum ascendat ad te, Domine: et descendat
super nos misericordia tua.

\textit{Deinde incensat altare, dicens}:

\biblia{Ps. 140, 2-4}

\initialis{D}irigatur, Domine, oratio mea, sicut incensum, in conspectu tuo:
elevatio manuum mearum sacrificium vespertinum.  Pone, Domine, custodiam ori
meo, et ostium circumstantiae labiis meis: ut non declinet cor meum in verba
malitiae, ad excusandas excusationes in peccatis.

\textit{Dum reddit thuribulum diacono, dicit}:

\initialis{A}ccendat in nobis Dominus ignem sui amoris et flammam aeternae
caritatis.  Amen.

\textit{Postea incensatur sacerdos a diacono, deinde alii per ordinem.}

\directio{stans ad incensionem}

\section{Lavatio}

\textit{Interim sacerdos lavat manus, dicens:}

\biblia{Ps. 25, 6-12}

\initialis{L}avabo inter innocentes manus meas: et circumdabo altare tuum,
Domine: Ut audiam vocem laudis, et enarrem universa mirabilia tua.  Domine,
dilexi decorem domus tuae, et locum habitationis gloriae tuae.  Ne perdas cum
impiis, Deus, animam meam, et cum viris sanguinum vitam meam: In quorum manibus
iniquitates sunt: dextera eorum repleta est muneribus.  Ego autem in innocentia
mea ingressus sum: redime me, et miserere mei.  Pes meus stetit in directo: in
ecclesiis benedicam te, Domine.  Gloria Patri, et Filio, et Spiritui Sancto.
Sicut erat in principio, et nunc, et semper: et in saecula saeculorum.  Amen.

\divisio

\textit{%
    In Missis defunctorum, et tempore Passionis in Missis de Tempore omittitur
    Gloria Patri.
}

\divisio

\vspace{-0.5\baselineskip}

\section{Sancta Trinitas}

\textit{%
    Deinde, aliquantulum inclinatus in medio altaris, iunctis manibus super eo,
    dicit:
}

\initialis{S}uscipe, sancta Trinitas, hanc oblationem, quam tibi offerimus ob
memoriam passionis, resurrectionis et ascensionis Iesu Christi Domini nostri: et
in honorem beatae Mariae semper Virginis, et beati Ioannis Baptistae, et
sanctorum Apostolorum Petri et Pauli, et istorum, et omnium Sanctorum: ut illis
proficiat ad honorem, nobis autem ad salutem: et illi pro nobis intercedere
dignentur in caelis, quorum memoriam agimus in terris.  Per eundem Christum
Dominum nostrum.  Amen.

\section{Orate Fratres}

\textit{%
    Postea osculatur altare et, versus ad populum, extendens et iungens manus,
    voce paululum elevata, dicit:
}

\initialis{O}rate, fratres: ut meum ac vestrum sacrificium acceptabile fiat apud
Deum Patrem omnipotentem.

\textit{Minister, seu circumstantes respondet: alioquin ipsemet sacerdos}:

\initialis{S}uscipiat Dominus sacrificium de manibus tuis (\textit{vel meis}) ad
laudem et gloriam nominis sui, ad utilitatem quoque nostram, totiusque Ecclesiae
suae sanctae.

\textit{Sacerdos submissa voce dicit:}

\sacerdos{Amen.}

\section{Secreta}

\textit{Deinde, manibus extensis, absolute sine} Oremus \textit{subiungit
orationes secretas.}

\proprius{Secreta}

\textit{Quibus finitis, cum pervenerit ad conclusionem, clara voce dicit}:

\versiculum{Per omnia saecula saeculorum}
\alternatim{\par\responsorium{Amen.}}

\textit{cum praefatione, ut in sequentibus}.

\section{Praefatio}

\directio{stans}

\textit{Praefationem incipit ambabus manibus positis hinc inde super altare:
quae aliquantulum elevat, cum dicit}: Sursum corda.  \textit{Iungit eas ante
pectus, et caput inclinat, cum dicit}: Gratias agamus Domino Deo nostro.
\textit{Deinde disiungit manus, et disiunctas tenet usque ad finem
praefationis:}

\versiculum{Dominus vobiscum.}
\alternatim{%
    \par\responsorium{Et cum spiritu tuo.}
    \par\versiculum{Susrum corda.}
    \par\responsorium{Habemus ad Dominum.}
    \par\versiculum{Gratias agamus Domino Deo nostro.}
    \par\responsorium{Dignum et iustum est.}
}

\proprius{Praefatio}

\initialis{V}ere dignum et iustum est, aequum et salutare, nos tibi semper, et
ubique gratias agere: Domine sancte, Pater omnipotens, aeterne Deus: Qui cum
unigenito Filio tuo, et Spiritu Sancto, unus es Deus, unus es Dominus: non in
unius singularitate personae, sed in unius Trinitate substantiae.  Quod enim de
tua gloria, relevante te, credimus, hoc de Filio tuo, hoc de Spiritu Sancto,
sine differentia discretionis sentimus.  Ut in confessione verae, sempiternaeque
Deitatis, et in personis proprietas, et in essentia unitas, et in maiestate
adoretur aequalitas.  Quam laudant Angeli atque Archangeli, Cherubim quoque ac
Seraphim: qui non cessant clamare quotidie, una voce dicens:

\section{Sanctus}

\directio{genuflectens}

\textit{qua finita, iterum iungit eas, et inclinatus dicit}: Sanctus.
\textit{Et cum dicit}: Benedictus qui venit, \textit{signum crucis sibi producit
a fronte ad pectus}.

\sinus

{
    \Large\centering
    \par Sanctus, sanctus, sanctus, Dominus Deus Sabaoth.  Plani sunt caeli et
    terra gloria tua.
    \par Hosanna in excelsis.
    \par Bene\cross{}dictus qui venit in nomine Domini.
    \par Hosanna in excelsis.
    \par
}

\section{Canon}

\textit{%
    Finita praefatione, sacerdos, extendes, elevans aliquantulum et iungens
    manus, elevansque ad caelum oculos, et statim demittens, profunde inclinatus
    ante altare, manibus super eo positis, dicit secreto:
}

\initialis{T}e igitur, clementissime Pater, per Iesum Christum, Filium tuum,
Dominum nostrum, supplices rogamus ac petimus (\textit{osculatur altare et,
iunctis manibus ante pectus dicit:}) uti accepta habeas et benedicas
(\textit{signat ter super hostiam et calicem simul, dicens:}) haec \cross{}
dona, haec \cross{} munera, haec \cross{} sancta sacrificia illibata
(\textit{extensis manibus prosequitur:}) in primis, quae tibi offerimus pro
Ecclesia tua sancta catholica: quam pacificare, custodire, adunare et regere
digneris toto orbe terrarum: una cum famulo tuo Papa nostro \textit{N.} et
Antistite nostro \textit{N.} et omnibus orthodoxis, atque catholicae et
apostolicae fidei cultoribus.

\subsection{Commemoratio pro vivis}

\initialis{M}emento, Domine, famulorum famularumque tuarum \textit{N.} et
\textit{N.} (\textit{iungit manus, orat aliquantulum pro quibus orare intendit:
deinde manibus extensis prosequitur:}) et omnium circumstantium, quorum tibi
fides cognita est et nota devotio, pro quibus tibi offerimus: vel qui tibi
offerunt hoc sacrificium laudis, pro se suisque omnibus: pro redemptione
animarum suarum, pro spe salutis et incolumitatis suae: tibique reddunt vota sua
aeterno Deo, vivo et vero.

\subsection{Infra Actionem}

\initialis{C}ommunicantes, et memoriam venerantes, in primis gloriosae semper
Virginis Mariae, Genetricis Dei et Domini nostri Iesu Christi: sed et beati
Ioseph, eiusdem Virginis Sponsi, et beatorum Apostolorum ac Martyrum tuorum,
Petri et Pauli, Andreae, Iacobi, Ioannis, Thomae, Iacobi, Philippi,
Bartholomaei, Matthaei, Simonis et Thaddaei: Lini, Cleti, Clementis, Xysti,
Cornelii, Cypriani, Laurentii, Chrysogoni, Ioannis et Pauli, Cosmae et Damiani:
et omnium Sanctorum tuorum; quorum meritis precibusque concedas, ut in omnibus
protectionis tuae muniamur auxilio.  (\textit{iungit manus}) Per eundem Christum
Dominum nostrum.  Amen.

\subsection{Hanc Igitur}

\textit{Tenens manus expansas super oblata, dicit:}

\sinus

\initialis{H}anc igitur oblationem servitutis nostrae, sed et cunctae familiae
tuae, quaesumus, Domine, ut placatus accipias: diesque nostros in tua pace
disponas, atque ab aeterna damnatione nos eripi, et in electorum tuorum iubeas
grege numerari.  (\textit{iungit manus}) Per Christum Dominum nostrum.  Amen.

\subsection{Quam Oblationem}

\initialis{Q}uam oblationem tu, Deus, in omnibus, quaesumus (\textit{signat ter
super oblata}) bene\cross{}dictam, adscrip\cross{}tam, ra\cross{}tam,
rationabilem, acceptabilemque facere digneris: (\textit{signat semel super
hostiam}) ut nobis Cor\cross{}pus, (\textit{et semel super calicem}) et
San\cross{}guis fiat dilectissimi Filii tui (\textit{iungit manus}) Domini
nostri Iesu Christi.

\initialis{Q}ui pridie quam pateretur (\textit{accipit hostiam}) accepit panem
in sanctas ac venerabiles manus suas: (\textit{elevat oculos ad caelum}) et
elevatis oculis in caelum ad te Deum Patrem suum omnipotentem (\textit{caput
inclinat}) tibi gratias agens, (\textit{signat super hostiam})
bene\cross{}dixit, fregit, deditque discipulis suis, dicens: Accipite, et
manducate ex hoc omnes.

\textit{%
    Tenens ambabus manibus hostiam inter indices et pollices, profert verba
    consecrationis distincte et attente super hostiam, et simul super omnes, si
    plures sint consecrandae.
}

\vspace{\baselineskip}

\biblia{Lc. 22, 19-20}

\vspace{0.5\baselineskip}

{\Large\centering Hoc est enim Corpus meum.\par}

\vspace{0.5\baselineskip}

\sinus\sinus\sinus

\vspace{\baselineskip}

\textit{%
    Quibus verbis prolatis, statim hostiam consecratam genuflexus adorat:
    surgit, ostendit populo, reponit super corporale, et genuflexus iterum
    adorat: nec amplius pollices et indices disiungit, nisi quando hostia
    tractanda est, usque ad ablutionem digitorum.  Tunc, detecto calice, dicit:
}

\initialis{S}imili modo postquam cenatum est (\textit{ambabus manibus accipit
calicem}), accipiens et hunc praeclarum calicem in sanctas ac venerabiles manus
suas: item (\textit{caput inclinat}) tibi gratias agens, (\textit{sinistra
tenens calicem, dextera signat super eum}) bene\cross{}dixit, deditque
discipulis suis, dicens: Accipite, et bibite ex eo omnes.

\textit{%
    Profert verba consecrationis super calicem attente et continuate, tenens
    illum parum elevatum.
}

{
    \Large\centering
    Hic est enim Calix Sanguinis mei, novi et aeterni testamenti: mysterium
    fidei: qui pro vobis et pro multis effundetur in remissionem peccatorum.
    \par
}

\sinus\sinus\sinus

\vspace{\baselineskip}

\textit{Quibus verbis prolatis, deponit calicem super corporale, et dicens:}

\vspace{\baselineskip}

{\Large\centering Haec quotiescumque feceritis, in mei memoriam facietis.\par}

\vspace{\baselineskip}

\textit{%
    genuflexus adorat: surgit, ostendit populo, deponit, cooperit, et genuflexus
    iterum adorat.  Deinde, disiunctis manibus, dicit:
}

\initialis{U}nde et memores, Domine, nos servi tui, sed et plebs tua sancta,
eiusdem Christi Filii tui, Domini nostri, tam beatae passionis, nec non et ab
inferis resurrectionis, sed et in caelos gloriosae ascensionis: offerimus
praeclarae maiestati tuae de tuis donis ac datis, (\textit{iungit manus, et
signat ter super hostiam et calicem simul, dicens:}) hostiam \cross{} puram,
hostiam \cross{} sanctam, hostiam \cross{} immaculata, (\textit{signat semel
super hostiam, dicens:}) Panem \cross{} sanctum vitae aeternae, (\textit{et
semel super calicem, dicens:}) et Calicem \cross{} salutis perpetuae.

\textit{Extensis manibus prosequitur:}

\initialis{S}upra quae propitio ac sereno vultu respicere digneris: et accepta
habere, sicuti accepta habere dignatus es munera pueri tui iusti Abel, et
sacrificium Patriarchae nostri Abrahae: et quod tibi obtulit summus sacerdos
tuus Melchisedech, sanctum sacrificium, immaculatam hostiam.

\textit{Profunde inclinatus, iunctis manibus et super altare positis, dicit:}

\initialis{S}upplices te rogamus, omnipotens Deus: iube haec perferri per manus
sancti Angeli tui in sublime altare tuum, in cospectu divinae maiestatis tuae:
ut, quotquot, (\textit{osculatur altare}) ex hac altaris participatione
sacrosanctum Filii tui (\textit{iungit manus, et signat semel super hostiam, et
semel super calicem}) Cor\cross{}pus et San\cross{}guinem sumpserimus
(\textit{seipsum signat, dicens:}) omni benedictione caelesti et gratia
repleamur.  (\textit{Iungit manus}) Per eundem Christum Dominum nostrum.  Amen.

\subsection{Commemoratio pro defunctis}

\initialis{M}emento etiam, Domine, famulorum famularumque tuarum \textit{N.} et
\textit{N.}, qui nos praecesserunt cum signo fidei, et dormiunt in somno pacis.

\textit{%
    Iungit manus, orat aliquantulum pro iis defunctis, pro quibus orare
    intendit, deinde extensis manibus prosequitur:
}

Ipsis, Domine, et omnibus in Christo quiescentibus, locum refrigerii, lucis et
pacis, ut indulgeas, deprecamur.  (\textit{Iungit manus et caput inclinat,
dicens:}) Per eundem Christum Dominum nostrum.  Amen.

\textit{Manu dextera percutit sibi pectus, elata aliquantulum voce dicens:}

\initialis{N}obis quoque peccatoribus (\textit{extensis manibus ut prius,
secrete prosequitur:}) famulis tuis, de multitudine miserationum tuarum
sperantibus, partem aliquam et societatem donare digneris, cum tuis sanctis
Apostolis et Martyribus: cum Ioanne, Stephano, Matthia, Barnaba, Ignatio,
Alexandro, Marcellino, Petro, Felicitate, Perpetua, Agatha, Lucia, Agnete,
Caecilia, Anastasia, et omnibus Sanctis tuis: intra quorum nos consortium, non
aestimator meriti, sed veniae, quaesumus, largitor admitte.  (\textit{Iungit
manus}) Per Christum Dominum nostrum.

\subsection{Doxologia finalis et elevatio minor}

\initialis{P}er quem haec omnia, Domine, semper bona creas, (\textit{signat ter
super hostiam et calicem simul, dicens:}) sancti\cross{}ficas,
vivi\cross{}ficas, bene\cross{}dicis et praestas nobis.

\textit{%
    Discooperit calicem, genuflectit, accipit hostiam inter pollicem et indicem
    manus dexterae: et tenens sinistra calicem, cum hostia signat ter a labio ad
    labium calicis, dicens:
}

Per ip\cross{}sum, et cum ip\cross{}so, et in ip\cross{}so, (\textit{cum ipsa
hostia signat bis inter se et calicem, dicens:}) est tibi Deo Patri \cross{}
omnipotenti, in unitate Spiritus \cross{} Sancti, (\textit{elevans parum calicem
cum hostia, dicit:}) omnis honor, et gloria.

\textit{%
    Reponit hostiam, calicem palla cooperit, genuflectit, surgit, et dicit
    intellegibili voce vel cantat:
}

\directio{stans}

\versiculum{Per omnia saecula saeculorum.}
\alternatim{\par\responsorium{Amen.}}

\subsection{Pater Noster}

\textit{Iungit manus.}

\versiculum{%
    Oremus.  Praeceptis salutaribus moniti, et divina instituitione formati,
    audemus dicere:
}

\textit{Extendit manus.}

\initialis{P}ater noster, qui es in caelis: sanctificetur nomen tuum: adveniat
regnum tuum: Fiat voluntas tua, sicut in caelo, et in terra.  Panem nostrum
cotidianum da nobis hodie: Et dimitte nobis debita nostra, sicut et nos
dimittimus debitoribus nostris.  Et ne nos inducas in tentationem.

\responsorium{Sed libera nos a malo.}

\textit{Sacerdos secrete dicit:}

\versiculum{Amen.}

\textit{%
    Deinde manu dextera accipit inter indicem et medium digitos patenam, quam
    tenens super altare erectam, dicit secrete:
}

\initialis{L}ibera nos, quaesumus Domine, ab omnibus malis, praeteritis,
prasentibus et futuris: et intercedente beata et gloriosa semper Virgine Dei
Genetrice Maria, cum beatis Apostolis tuis Petro et Paulo, atque Andrea, et
omnibus Sanctis, (\textit{signat se cum patena a fronte ad pectus}) da propitius
pacem in diebus nostris: (\textit{patenam osculatur}) ut, ope misericordiae tuae
adiuti, et a peccato simus semper liberi et ab omni perturbatione securi.

\subsection{Fracto Panis}

\textit{%
    Submittit patenam hostiae, discooperit calicem, genuflectit, surgit, accipit
    hostiam, et eam super calicem tenens utraque manu, frangit per medium,
    dicens:
}

{\Large\centering Per eundem Dominum nostrum Iesum Christum, Filium tuum.\par}

\textit{%
    Et mediam partem, quam in dextera manu tenet, ponit super patenam.  Deinde
    ex parte, quae in sinistra remanserat, frangit particulam, dicens:
}

{
    \Large\centering
    Qui tecum vivit et regnat in unitate Spiritus Sancti Deus.
    \par
}

\textit{%
    Aliam mediam partem, quam in sinistra manu habet, adiungit mediae super
    patenam positae, et particulam parvam dextera retinens super calicem, quem
    sinistra per nodum infra cuppam tenet, dicit intellegibili voce vel cantat:
}

\versiculum{Per omnia saecula saeculorum.}
\alternatim{\par\responsorium{Amen.}}

\subsection{Commixtio Corporis et Sanguinis}

\textit{Cum ipsa particula signat ter super calicem, dicens:}

\versiculum{Pax \cross{} Domini sit \cross{} semper vobis\cross{}cum.}
\alternatim{\par\responsorium{Et cum spiritu tuo.}}

\textit{Particulam ipsam immittit in calicem, dicens secrete:}

\initialis{H}aec commixtio, et consecratio Corporis et Sanguinis Domini nostri
Iesu Christi, fiat accipientibus nobis in vitam aeternam.  Amen.

\section{Agnus Dei}

\directio{genuflectens}

\textit{%
    Cooperit calicem, genuflectit, surgit, et inclinatus Sacramento, iunctis
    manibus, et per pectus percutiens, intellegibili voce dicit:
}

{
    \Large\centering
    \par Agnus Dei, qui tollis peccata mundi: miserere nobis.
    \par Agnus Dei, qui tollis peccata mundi: miserere nobis.
    \par Agnus Dei, qui tollis peccata mundi: dona nobis pacem.
    \par
}

\divisio

\textit{In Missis defunctorum non dicitur} miserere nobis, \textit{sed eius
loco} dona eis requiem, \textit{et in tertio additur} sempiternam.

\divisio

\textit{%
    Deinde, iunctis manibus super altare, inclinatus dicit secrete sequentes
    orationes:
}

\initialis{D}omine Iesu Christe, qui dixisti Apostolis tuis: Pacem relinquo
vobis, pacem meam do vobis: ne respicias peccata mea, sed fidem Ecclesiae tuae;
eamque secundum voluntatem tuam pacificare et coadunare digneris: Qui vivis et
regnas Deus per omnia saecula saeculorum.  Amen.

\textit{Si danda est pax, osculatur altare et, dans pacem, dicit:}

\versiculum{Pax tecum.}
\alternatim{\par\responsorium{Et cum spiritu tuo.}}

\divisio

\textit{In Missis defunctorum non datur pax, neque dicitur praecedens oratio.}

\divisio

\initialis{D}omine Iesu Christe, Fili Dei vivi, qui ex voluntate Patris,
cooperante Spiritu Sancto, per mortem tuam mundum vivificasti: libera me per hoc
sacrosanctum Corpus et Sanguinem tuum ab omnibus iniquitatibus meis et universis
malis: et fac me tuis semper inhaerere mandatis, et a te numquam separari
permittas: Qui cum eodem Deo Patre et Spiritu Sancto vivis et regnas Deus in
saecula saeculorum.  Amen.

\initialis{P}erceptio Corporis tui, Domine Iesu Christe, quod ego indignus
sumere praesumo, non mihi proveniat in iudicium et condemnationem: sed pro tua
pietate prosit mihi ad tutamentum mentis et corporis, et ad medelam
percipiendam: Qui vivis et regnas cum Deo Patre in unitate Spiritus Sancti Deus,
per omnia saecula saeculorum.  Amen.

\section{Communio}

\textit{Genuflectit, surgit, et dicit:}

\initialis{P}anem caelestem accipiam, et nomen Domini invocabo.

\textit{%
    Deinde parum inclinatus, accipit ambas partes hostiae inter pollicem et
    indicem sinistrae manus, et patenam inter eundem indicem et medium supponit,
    et dextera tribus vicibus percutiens pectus, elata aliquantulum voce, ter
    dicit devote et humiliter:
}

\sinus\sinus\sinus

\biblia{Mt. 8, 8}

\initialis{D}omine, non sum dignus (\textit{et secrete prosequitur:}) ut intres
sub tectum meum: sed tantum dic verbo, et sanabitur anima mea.

\subsection{Communio Fidelis}

\textit{Postea, dextera se signans cum hostia super patenam, dicit:}

\initialis{C}orpus Domini nostri Iesu Christi custodiat animam meam in vitam
aeternam.  Amen.

\textit{%
    Et, se inclinans, reverenter sumit ambas partes hostiae: quibus sumptis,
    deponit patenam super corporale, et erigens se iungit manus, et quiescit
    aliquantulum in meditatione sanctissimi Sacramenti.  Deinde discooperit
    calicem, genuflectit, colligit fragmenta, si quae sint, extergit patenam
    super calicem, interim dicens:
}

\biblia{Ps. 115, 12-13}

\initialis{Q}uid retribuam Domino pro omnibus quae retribuit mihi?  Calicem
salutaris accipiam, et nomen Domini invocabo.

\biblia{Ps. 18, 3}

Laudans invocabo Dominum, et ab inimicis meis salvus ero.

\textit{Accipit calicem manu dextera et, eo se signans, dicit:}

\initialis{S}anguis Domini nostri Iesu Christi custodiat animam meam in vitam
aeternam.  Amen.

\textit{%
    Et, sinistra supponens patenam calici, reverenter sumit totum Sanguinem cum
    particula.  Quo sumpto, si qui sunt communicandi, eos communicet, antequam
    se purificet.
}

\versiculum{Ecce Agnus Dei, ecce qui tollit peccata mundi.}
\alternatim{%
    \par\responsorium{%
        Domine, non sum dignis, ut intres sub tectum meum: sed tantum dic verbo,
        et sanabitur anima mea.
    }
}

\initialis{C}orpus Domini nostri Iesu Christi custodiat animam tuam in vitam
aeternam.

\vspace{0.5\baselineskip}\responsorium{Amen.}

\textit{Postea dicit:}

\initialis{Q}uod ore sumpsimus, Domine, pura mente capiamus: et de munere
temporali fiat nobis remedium sempiternum.

\textit{%
    Interim porrigit calicem ministro, qui infundit in eo parum vini, quo se
    purificat: deinde prosequitur:
}

\initialis{C}orpus tuum, Domine, quod sumpsi, et Sanguis, quem potavi, adhaereat
visceribus meis: et praesta; ut in me non remaneat scelerum macula, quem pura et
sancta refecerunt sacramenta: Qui vivis et regnas in saecula saeculorum.  Amen.

\subsection{Oratio Communionis}

\textit{%
    Abluit et extergit digitos, ac sumit ablutionem: extergit os et calicem,
    quem, plicato corporali, operit et collocat in altari ut prius: deinde
    prosequitur Missam.
}

\proprius{Communio}

\versiculum{Dominus vobiscum.}
\alternatim{%
    \par\responsorium{Et cum spiritu tuo.}
    \par\versiculum{Oremus.}
}

\section{Post Communionem}

\directio{stans}

\proprius{Post Communionem}

\textit{Dicto, post ultimam orationem,}

\versiculum{Dominus vobiscum.}
\alternatim{\par\responsorium{Et cum spiritu tuo.}}

\textit{dicit}

\versiculum{Ite, missa est.}

\divisio

\textit{Vel, si qua liturgica processio sequatur:}

\versiculum{Benedicamus Domino.}

\divisio

\responsorium{Deo gratias.}

\divisio

\textit{In Missis defunctorum dicit:}

\versiculum{Requiescant in pace}
\alternatim{\par\responsorium{Amen.}}

\divisio

\textit{%
    Tunc celebrans inclinat se ante medium altaris, et, manibus iunctis super
    illud, dicit secrete:
}

\initialis{P}laceat tibi, sancta Trinitas, obsequium servitutis meae: et
praesta: ut sacrificium, quod oculis tuae maiestatis indignus obtuli, tibi sit
acceptabile, mihique et omnibus, pro quibus illud obtuli, sit, te miserante,
propitiabile.  Per Christum Dominum nostrum.  Amen.

\subsection{Benedictio}

\textit{%
    Deinde osculatur altare: et elevatis oculis, extendens, elevans et iungens
    manus caputque Cruci inclinans, dicit:
}

\versiculum{Benedicat vos omnipotens Deus.}

\textit{%
    Et versus ad populum, semel tantum benedicens, etiam in Missis solemnibus,
    prosequitur:
}

\versiculum{Pater, et Filius, \cross{} et Spiritus Sanctus.}
\alternatim{\par\responsorium{Amen.}}

\divisio

\textit{In Missa pontificali ter benedicitur, ut in Pontificali habetur.}

\textit{In Missis in quibus dictum est} Benedicamus Domino \textit{vel}
Requiescant in pace, \textit{non datur benedictio, sed dicto} Placeat \textit{et
osculato altari, celebrans legit, si dicendum sit, Evangelium S. Ioannis.}

\divisio

\subsection{Evangelium Secundum Ioannem}

\directio{stans}

\textit{Deinde sacerdos in latere Evangelii, iunctis manibus dicit:}

\versiculum{Dominus vobiscum.}
\alternatim{\par\responsorium{Et cum spiritu tuo.}}

\textit{%
    Et signans signo crucis primum altare vel librum, deinde se in fronte, ore
    et pectore, dicit:
}

{{\liturgicalfont\fontsize{48}{48}\selectfont\raisebox{-0.3em}{᛭}}}
Initium sancti Evangelii secundum Ioannem.

\textit{Vel si aliud Evangelium legendum sit}: Sequentia sancti Evangelii
\textit{etc.}

\responsorium{Gloria tibi, Domine.}

\textit{Iunctis manibus prosequitur:}

\pagebreak

\biblia{Io 1, 1-14}

\initialis{I}n principio erat Verbum, et Verbum erat apud Deum, et Deus erat
Verbum.  Hoc erat in principio apud Deum.  Omnia per ipsum facta sunt: et sine
ipso factum est nihil, quod factum est: in ipso vita erat, et vita erat lux
hominum: et lux in tenebris lucet, et tenebrae eam non comprehenderunt.

Fuit homo missus a Deo, cui nomen erat Ioannes.  Hic venit in testimonium, ut
testimonium perhiberet de lumine, ut omnes crederent per illum.  Non erat ille
lux, sed ut testimonium perhiberet de lumine.

Erat lux vera, quae illuminat omnem hominem venientem in hunc mundum.  In mundo
erat, et mundus per ipsum factus est, et mundus eum non cognovit.  In propria
venit, et sui eum non receperunt.  Quotquot autem receperunt eum, dedit eis
potestatem filios Dei fieri, his, qui credunt in nomine eius: qui non ex
sanguinibus, neque ex voluntate carnis, neque ex voluntate viri, sed ex Deo nati
sunt.  (\textit{Genuflectit dicens:}) \textsc{Et Verbum caro factum est}
(\textit{et surgens prosequitur:}) et habitavit in nobis: et vidimus gloriam
eius, gloriam quasi Unigeniti a Patre, plenum gratiae et veritatis.

\responsorium{Deo gratias.}
